\subsection*{Introduction}

This project allows to address the G\+PU on Android devices using the Open\+CL framework. In contrast to other projects the Open\+CL library of the device is loaded only at runtime. There is no need to include it during the build process.

Two data mining algorithms (D\+B\+S\+C\+AN and Kmeans) have been implemented with two different programming languages and different programming paradigms (single-\/threaded, multi-\/threaded, task/data parallelism (G\+PU)).

\subsection*{Docs}

There is a \href{app/doc/html/index.html}{\tt html} and \href{app/doc/latex/refman.pdf}{\tt pdf} doxygen documentation available for this project. As of 9/12/2021 only the C part (except dbscan\+\_\+c.\+c and kmeans\+\_\+c.\+c) has been included in this documentation. Additional documentation will be available soon.

\subsection*{Installation}

Clone this repository (either with \char`\"{}git clone\char`\"{} or downloading and extracting the zip-\/file). Import the project to Android\+Studio. Attach your Android device and enable developper options (see manual of your device). Build and run this project on your device. It should run out of the box.

\subsection*{Setup}

This app has only one activity. The data mining jobs are executed in a deferred manner in the background. If the devices is rebooted, the calculations will resume automatically. When the user launches the app, the main activity tries to connect to the background jobs and tries to read out the status information. If there are no background jobs, the user can submit a new background job. If there are already background jobs, the status is displayed and the user can cancel the jobs.

The app tries to find the Open\+CL library on the device automatically. If an Open\+CL library is found and can be loaded, some information is displayed. If not, the user can enter the path to the Open\+CL library on the device manually and try to load it manually.

\subsection*{How to set parameters}

The parameters for the data mining jobs have to be set in the file app/src/main/res/values/values.\+xml. The following attributes can be set\+:


\begin{DoxyItemize}
\item {\bfseries mode} (string)\+:
\begin{DoxyItemize}
\item \char`\"{}dynamic\char`\"{} multiple test are made with built in values. The attributes \textquotesingle{}clusterno\textquotesingle{}, \textquotesingle{}clustersize\textquotesingle{} and \textquotesingle{}features\textquotesingle{} must be set correctly but will not be used.
\item \char`\"{}fixed\char`\"{} tests are made with the attributes specified in this file
\end{DoxyItemize}
\item {\bfseries passes} (integer) Number of passes that should be made
\item {\bfseries threads} (integer) Number of threads that should be used for multithreaded implementations. Zero means the maximum number of available cores.
\item {\bfseries export} (boolean) \char`\"{}true\char`\"{} export results, \char`\"{}false\char`\"{} do not export results
\item {\bfseries append} (boolean) \char`\"{}true\char`\"{} append to old results if available, \char`\"{}false\char`\"{} delete old results
\item {\bfseries resultfilename} (string) Name of the csv-\/file in which to store the results. {\bfseries DO N\+OT P\+R\+E\+P\+E\+ND A P\+A\+T\+H!} The correct path will be prepended automatically. In the version for larger screen sizes the full path and name are shown in the info box.
\item {\bfseries log} (boolean) \char`\"{}true\char`\"{} log information, \char`\"{}false\char`\"{} do not log information (there is just one log-\/level)
\item {\bfseries logfilename} (string) Name of the text-\/file in which to store the results. {\bfseries DO N\+OT P\+R\+E\+P\+E\+ND A P\+A\+T\+H!} The correct path will be prepended automatically. In the version for larger screen sizes the full path and name are shown in the info box.
\item {\bfseries kmeanseps} Maximum cluster center displacement. If the sum of the absolute values of the cluster displacements drops below this threshold, the algorithm terminates.
\item {\bfseries dbscaneps} Search radius for D\+B\+S\+C\+AN (0=sqrt(features))
\item {\bfseries dbscanneigh} Minimum number of neighbours within the serach radius (0=10$\ast$features)
\item {\bfseries clusterno} Number of clusters to generate randomly. For Kmeans also the number of clusters to search for.
\item {\bfseries clustersize} Size of the clusters (equal size for all clusters).
\item {\bfseries features} Number of features for each data item.
\end{DoxyItemize}

In a future version an additional activity, that will allow to set the attributes on the device during runtime, will be added to this project. ~\newline
 \subsection*{Results}

The results are stored in csv-\/format on the device. The path of the app is used (should be something like sdcard/\+Android/data/com.\+example.\+dmocl/files). Log information is also stored in this path. This path is set automatically -\/ do not prepend the path in the values.\+xml file.

The result file has 21 columns separated by \textquotesingle{};\textquotesingle{}. The first four contain the parameters of the test (cores;size;cluster;features). The next five show the wall clock time for different implementations of the kmeans algorithm (Java, C, C+\+G\+PU, multithreaded Java, multithreaded C). The following five columns hold the wall clock times for the D\+B\+S\+C\+AN algorithm (again Java, C, C+\+G\+PU, multithreaded Java, multithreaded C). The next columns contain the exclusive time for the G\+PU and the multithreaded implementations. \textquotesingle{}Exclusive time\textquotesingle{} means the time needed for the execution of the entire algorithm minus the time needed for the setup of the G\+PU or the threads. Three values are collected by Kmeans and another three by D\+B\+S\+C\+AN. The last column is zero if the output of all the implementations of the D\+B\+S\+C\+AN algorithm was E\+X\+A\+C\+T\+LY equal. For Kmeans a comparison is not possible because each implementation selects the cluster centers randomly at the beginning. 